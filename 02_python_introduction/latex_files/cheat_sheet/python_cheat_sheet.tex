\documentclass[12pt]{article}

\usepackage[T1]{fontenc}
\usepackage{ascii}
\usepackage{titlesec}
\usepackage[top=5mm,bottom=0mm,footskip=0mm, headsep=0mm, margin=0.5in]{geometry}
\usepackage{xcolor}
\usepackage{lmodern}
\usepackage{upquote}
\usepackage{longtable}
%\usepackage{layout}
\renewcommand*\familydefault{\ttdefault}


\titlespacing*{\section}
{0cm}{0.8cm}{0.3cm}
\titlespacing*{\subsection}
{0cm}{0.7cm}{0.4cm}

\titleformat*{\section}{\Large\bfseries\color{cyan}}
\titleformat*{\subsection}{\large\bfseries\sf}

%\setlength{\voffset}{-0.75in}
%\setlength{\headsep}{5pt}
%\setlength{\footskip}{0pt}
%\setlength{\footskip}{0pt}

\begin{document}
%\fontseries{lc}\selectfont

\pagenumbering{gobble}
%\lstset{language=Python, basicstyle=\ttfamily\small}
\noindent\textsc{\textbf{\Huge Python Cheat Sheet}}
\section*{Data types}

\subsection*{Types}
\begin{tabular}[l]{p{3.5cm} p{4cm} p{10cm}}
\textbf{Name} & \textbf{Python type} & \textbf{Examples} \\[0.2cm]
\textsf{integer} & int & 2; -45; 0 \\
\textsf{floating point} & float & 1.014; -12.64; 0.0 \\
\textsf{string} & str & \textquotesingle b\textquotesingle ; \textquotesingle banana\textquotesingle ; \textquotesingle banana cake\textquotesingle  \\
\textsf{boolean} & bool & True; False \\
\textsf{list} & list & [1, 2, 3]; [-1, \textquotesingle a\textquotesingle , True] \\
\textsf{tuple} & tuple & (1, 2, 3); (-1, \textquotesingle a\textquotesingle , True) \\
\textsf{dictionary} & dict & \{\textquotesingle banana\textquotesingle : 3, \textquotesingle pear\textquotesingle : 42, \textquotesingle alien fruit\textquotesingle : 0\}\\
\end{tabular}

%\vspace{0.5cm}
%\noindent\footnotesize(\emph{\*Note:} elements in \texttt{tuple} and \texttt{str} cannot be altered (\emph{immutability}) )

\subsection*{Type functions}
\begin{longtable}{p{3cm} p{8cm} p{7cm}}
%\begin{tabular}[l]{p{3cm} p{8cm} p{7cm}}
\textsf{\large\color{gray} list} & & \\[0.2cm]
\textbf{Function} & \textbf{Example} & \textbf{Explanation}\\[0.2cm]
[] & fruits = ["apple",  "grape", "kiwi] & create an list of strings \\
x.append(y) & [1, 2].append(3) \boldmath$\rightarrow$ [1, 2, 3] & \textsf{add y to end of x} \\
x.pop(y) & [1, 2].pop(1) \boldmath$\rightarrow$ 2 & \textsf{remove element at position y from x} \\
x.reverse() & [1, 2].reverse() \boldmath$\rightarrow$ [2, 1] & \textsf{reverse order of x} \\
x.sort() & [1, 2, 0].sort() \boldmath$\rightarrow$ [0, 1, 2] & \textsf{sort x} \\
x.count(y) & [1, 2, 1].count(1) \boldmath$\rightarrow$ 2 & \textsf{count occurrences of y in x}\\
y in x & 0 in [1, 2] \boldmath$\rightarrow$ False & \textsf{check if y is in x} \\
x + y & [1, 2] + [3, 4] \boldmath$\rightarrow$ [1, 2, 3, 4] & \textsf{concatenate x and y} \\[0.8cm]

\textsf{\large\color{gray} string} & & \\[0.2cm]
\textbf{Function} & \textbf{Example} & \textbf{Explanation}\\[0.2cm]
"" & name = "Rainer Zufall" & create a string \\
x.startswith(y) & \textquotesingle banana\textquotesingle .startswith(\textquotesingle ban\textquotesingle ) \boldmath$\rightarrow$ True & \textsf{check if x begins with y} \\
x.endswith(y) & \textquotesingle banana\textquotesingle .endswith(\textquotesingle ana\textquotesingle ) \boldmath$\rightarrow$ True & \textsf{check if x ends with y} \\
x.replace(y, z) & \textquotesingle banana\textquotesingle .replace(\textquotesingle a\textquotesingle , \textquotesingle o\textquotesingle ) \boldmath$\rightarrow$ \textquotesingle bonono\textquotesingle  & \textsf{replace every y with z} \\
y in x & \textquotesingle anan\textquotesingle\ in \textquotesingle banana\textquotesingle  \boldmath$\rightarrow$ True & \textsf{check if y is in x} \\
x.join(y) & \textquotesingle  \textquotesingle .join([\textquotesingle First\textquotesingle , \textquotesingle Last\textquotesingle ]) \boldmath$\rightarrow$ \textquotesingle First Last\textquotesingle  & \textsf{combine elements in y with string x} \\
x.lower() & \textquotesingle BaNaNa\textquotesingle .lower() \boldmath$\rightarrow$ \textquotesingle banana\textquotesingle  & \textsf{convert x to lower case} \\
x.upper() & \textquotesingle BaNaNa\textquotesingle .upper() \boldmath$\rightarrow$ \textquotesingle BANANA\textquotesingle  & \textsf{convert x to upper case} \\
x.split(y) & \textquotesingle ba na na\textquotesingle .split(\textquotesingle  \textquotesingle ) \boldmath$\rightarrow$ [\textquotesingle ba\textquotesingle , \textquotesingle na\textquotesingle , \textquotesingle na\textquotesingle ] & \textsf{list of substrings in x separated by y} \\
x.strip() & \textquotesingle   banana\textbackslash n\textquotesingle .strip() \boldmath$\rightarrow$ \textquotesingle banana\textquotesingle  & \textsf{remove leading and trailing whitespace} \\
x + y & \textquotesingle ban\textquotesingle  + \textquotesingle anana\textquotesingle  \boldmath$\rightarrow$ \textquotesingle banana\textquotesingle  & \textsf{concatenate x and y} \\[0.8cm]

\textsf{\large\color{gray} dictionary} & & \\[0.2cm]
\textbf{Function} & \textbf{Example} & \textbf{Explanation}\\[0.2cm]
\{\} & id\_2\_name = \{\} & \textsf{create a new empty dictionary} \\
set & id\_2\_name[\textquotesingle 9606\textquotesingle] = \textquotesingle Homo sapiens\textquotesingle & \textsf{add a new key value pair to the dict}\\
get & id\_2\_name[\textquotesingle 9606\textquotesingle] \boldmath$\rightarrow$ \textquotesingle Homo sapiens\textquotesingle & \textsf{access dictionary element using the key \textquotesingle 9606\textquotesingle to get its value}\\
update & id\_2\_name[\textquotesingle 9606\textquotesingle] = \textquotesingle Homo sapiens Linnaeus, 1758\textquotesingle & \textsf{overwrite existing value for given key}\\
delete & del id\_2\_name[\textquotesingle 9606\textquotesingle] & \textsf{remove the entry with key \textquotesingle 9606\textquotesingle}\\
%\end{tabular}\\[0.2cm]
\end{longtable}\\[0.2cm]


\subsection*{Slicing}
\begin{tabular}[l]{ l l l }
\textbf{Operation} & \textbf{Example} & \textbf{Explanation}\\[0.2cm]
x[y] & [1, 2, 3, 4, 5][0] \boldmath$\rightarrow$ 1 & \textsf{get single element} \\
x[y:z] & [1, 2, 3, 4, 5][1:3] \boldmath$\rightarrow$ [2, 3] & \textsf{get elements from positions y to z-1} \\
x[y:] & [1, 2, 3, 4, 5][2:] \boldmath$\rightarrow$ [3, 4, 5] & \textsf{get elements from positions y to the end} \\
x[y:-1] & [1, 2, 3, 4, 5][2:-1] \boldmath$\rightarrow$ [3, 4] & \textsf{get elements from positions y to the second last} \\
\end{tabular}

\section*{Operators}
\subsection*{Arithmetic operators}
%\begin{tabular}[c]{ l l l l }
\begin{longtable}[c]{ l l l l }
\textbf{Name} & \textbf{Operator} & \textbf{Example} & \textbf{Result}\\[0.2cm]
\textsf{Addition} & + & 9 + 2 & 11 \\
\textsf{Substraction} & -  & 9 - 2 & 7 \\
\textsf{Multiplication} & * & 9 * 2 & 18 \\
\textsf{Integer division} & // & 9 // 2 & 4 \\
\textsf{Float division} & / & 9 / 2 & 4.5 \\
\textsf{Exponent/Power} & ** & 9 ** 2 & 81 \\
\textsf{Modulus} & \% & 9 \% 2 & 1 \\
%\end{tabular}
\end{longtable}

\vspace{0.5cm}
\small{\noindent(\emph{\*Note:} +, -, *, / and ** have equivalents for direct assignment: +=, -=, *=, /= **=, \\
 e.g. x += 5 is equivalent to x = x + 5 )}

\subsection*{Boolean operators}
%\begin{tabular}[c]{ l l l l }
\begin{longtable}[c]{ l l l l }
\textbf{Name} & \textbf{Operator} & \textbf{Example} & \textbf{Result}\\[0.2cm]
\textsf{Equal} & == & 9 == 2 & False \\
\textsf{Not equal} & != & 9 != 2 & True \\
\textsf{Larger} & >  & 9 > 2 & True \\
\textsf{Larger equal} & >=  & 9 >= 2 & True \\
\textsf{Smaller} & < & 9 < 2 & False \\
\textsf{Smaller equal} & <= & 9 <= 2 & False \\
\textsf{AND} & and & (9 > 2) and (9 < 11) & True \\
\textsf{NOT} & not & not (9 > 2) & False \\
\textsf{OR} & or & (9 > 2) or (9 == 2) & True \\
%\end{tabular}
\end{longtable}

\section*{Control flow (if, for, while)}

\begin{minipage}[t]{0.40\textwidth}
\subsection*{If statement}
\begin{verbatim}
if  9 > 2:
    print('9 bigger than 2' )
elif 9 < 2:
    print('9 smaller than 2')
else:
    print('9 equals 2')
\end{verbatim}
\end{minipage}
\begin{minipage}[t]{0.30\textwidth}
\subsection*{For loop}
\begin{verbatim}
for i in [1, 2, 3, 4]:
    print(i)
\end{verbatim}
\end{minipage}
%\vspace{1cm}
\begin{minipage}[t]{0.25\textwidth}
\subsection*{While loop}
\begin{verbatim}
i = 1
while i <= 4:
    print(i)
    i += 1
\end{verbatim}
\end{minipage}

\section*{Functions and modules}
%\begin{tabular}[l]{p{3.7cm} p{8cm} p{6.5cm}}
\begin{longtable}[l]{p{3.7cm} p{8cm} p{6.5cm}}
\textsf{\large general} & & \\[0.2cm]
\textbf{Function} & \textbf{Example} & \textbf{Explanation}\\[0.2cm]
import x & import(math) & \textsf{import module x for usage}\\
from x import y & from math import sqrt & \textsf{import only x from module y} \\
print(x) & print(\textquotesingle hello world\textquotesingle ) & \textsf{show x on screen} \\
help(x) & help(math) & \textsf{show documentation for x} \\
exit() & exit() & exit python \\
type(x) & type(myDNA) \boldmath$\rightarrow$ string& type of x\\
len(x) & len([1, 2, 3, 4]) \boldmath$\rightarrow$ 4 & \textsf{length of x} \\
max(x, y) & max(1, 2) \boldmath$\rightarrow$ 2 & \textsf{the larger of two objects (also: min())} \\
abs(x) & abs(-1) \boldmath$\rightarrow$ 1 & \textsf{absolute value of x} \\
range(x, y) & range(1, 4) \boldmath$\rightarrow$ [1, 2, 3] & \textsf{list over numbers from x to y-1} \\
round(x, y) & round(1.205, 2) \boldmath$\rightarrow$ 1.21 & \textsf{round x to y decimals} \\[0.4cm]

\textsf{\large math} & & \\[0.2cm]
\textbf{Function} & \textbf{Example} & \textbf{Explanation}\\[0.2cm]
math.sqrt(x) & math.sqrt(4) \boldmath$\rightarrow$ 2 & \textsf{square root of x} \\
math.ceil(x) & math.ceil(0.9) \boldmath$\rightarrow$ 1.0 &  \textsf{round x up} \\
math.floor(x) & math.floor(0.9) \boldmath$\rightarrow$ 0.0 &  \textsf{round x down} \\
math.cos(x) & math.cos(0) \boldmath$\rightarrow$ 1.0  & \textsf{cosine of x (analogous: math.tan(x), math.sin(x))} \\[0.4cm]

\textsf{\large sys} & & \\[0.2cm]
\textbf{Function} & \textbf{Example} & \textbf{Explanation}\\[0.2cm]
sys.argv & \verb|['my_script.py' , 'my_dna.fasta']| & \textsf{parameters with which the script  called} \\
sys.path & \verb|['/Library/Python/2.7/site-packages']| & \textsf{directories in which python searches for modules} \\[0.4cm]

\textsf{\large re} & & \\[0.2cm]
\textbf{Function} & \textbf{Example} & \textbf{Explanation}\\[0.2cm]
re.search(exp, x) & re.search(\textquotesingle xy\textquotesingle , \textquotesingle banana\textquotesingle ) \boldmath$\rightarrow$ None & \textsf{check if regex exp occurs in string x}\\
re.search(exp, x).group(0) & re.search(\textquotesingle ana\textquotesingle , \textquotesingle banana\textquotesingle ) \boldmath$\rightarrow$ \textquotesingle ana\textquotesingle  & \textsf{check if exp in x and show pattern found}\\
re.findall(exp, x) & re.findall(\textquotesingle ta\textquotesingle , \textquotesingle tautax\textquotesingle ) \boldmath$\rightarrow$ [\textquotesingle ta\textquotesingle , \textquotesingle ta\textquotesingle ] & \textsf{find all occurences of exp in x}\\
\verb$[y|z]x$ & \textquotesingle \verb$[A|a]lex$\textquotesingle  \boldmath$\rightarrow$ \textquotesingle Alex\textquotesingle ; \textquotesingle alex\textquotesingle  & \textsf{hits substrings starting with y or z, followed by x}\\
xy* & \textquotesingle TA*\textquotesingle  \boldmath$\rightarrow$ \textquotesingle T\textquotesingle ; \textquotesingle TA\textquotesingle , \textquotesingle TAA\textquotesingle , .. & \textsf{hits substrings starting with x followed by} \\
 & & \textsf{any number of y\textquotesingle s} \\
\textbackslash w & \textquotesingle \textbackslash wno\textquotesingle  \boldmath$\rightarrow$ \textquotesingle Ano\textquotesingle ; \textquotesingle zno\textquotesingle , \textquotesingle 9no\textquotesingle , \textquotesingle \_no\textquotesingle , .. & \textsf{hits any letter or number 0-9 (upper/lower case) and \_}\\
\textbackslash d & \textquotesingle \textbackslash d015\textquotesingle  \boldmath$\rightarrow$ \textquotesingle 0015\textquotesingle ; \textquotesingle 2015\textquotesingle , .. & \textsf{hits any number 0-9}\\
\textbackslash s & \textquotesingle Mr\textbackslash s\textquotesingle  \boldmath$\rightarrow$ \textquotesingle Mr \textquotesingle ; \textquotesingle Mr\textbackslash n\textquotesingle , \textquotesingle Mr\textbackslash t\textquotesingle  & \textsf{hits any white space} \\
%\end{tabular}
\end{longtable}

\section*{Reading and writing files}
\begin{minipage}[t]{0.60\textwidth}

\subsection*{Reading}
\begin{verbatim}
#open file in reading mode
my_file = open('my_dna.fasta', 'r')

#read a single line
a_line = my_file.readline()

#iterate over all lines in a file
for line in my_file:
	...
#read all lines
all_lines = my_file.readlines()

#close the file
my_file.close()
\end{verbatim}
\end{minipage}
\begin{minipage}[t]{0.40\textwidth}
\subsection*{Writing}
\begin{verbatim}
#open file in writing mode
my_file = open('my_dna.fasta', 'w')

#write string x into file
my_file.write(x)

#write all strings in list x into file
my_file.writelines(x)

#close the file
my_file.close()
\end{verbatim}
\end{minipage}


\section*{IPython magics}
\begin{minipage}[t]{0.60\textwidth}
\subsection*{Using other programming languages}
\begin{verbatim}
R_results = %R <R command>
%html
%sql
%ruby
%cython
%python2
...
\end{verbatim}
\subsection*{Profiling your code}

\%timeit <python command>

	\textsf{-- returns average time your command took to
	run, useful when you want to optimize speed}
\vspace{1cm}

\%prun/lprun -e <function name> <python command>

	\textsf{-- tells you the percentage of time your
	command spent in each line (lprun)
	or each function (prun), helps you
	narrow down performance critical
	parts in your code}



\end{minipage}
\begin{minipage}[t]{0.40\textwidth}
\subsection*{Debugging}

\%debug\\
	\textsf{-- enter pdb (python debugger), in which you
	 can navigate and inspect variables that
	 produced errors}

\end{minipage}

\end{document}
